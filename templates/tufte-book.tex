\hypersetup{colorlinks}% uncomment this line if you prefer colored hyperlinks (e.g., for onscreen viewing)
\usepackage{amsfonts, soul, microtype}

% toc
%\usepackage{titletoc}
%

% section and subsection headings
\usepackage{titlesec}

\titleformat{\section}
  {\normalfont\fontsize{12}{15}\bfseries}{\thesection}{1em}{}

\usepackage[linesnumbered, ruled]{algorithm2e}
\usepackage{algorithmic}
\usepackage[parfill]{parskip}

%\usepackage{thmtools}
%\theoremstyle{definition}
\newtheorem{theorem}{Theorem}[chapter]
\newtheorem{lemma}[theorem]{Lemma}
\newtheorem{corollary}[theorem]{Corollary}
\newtheorem{proposition}[theorem]{Proposition}
\newtheorem{property}[theorem]{Property}
\newtheorem{fact}[theorem]{Fact}
\newtheorem{problem}[theorem]{Problem}
\newtheorem{exercise}[theorem]{Exercise}
\newtheorem{example}[theorem]{Example}
\newtheorem{definition}[theorem]{Definition}
\newtheorem{remark}[theorem]{Remark}
\newtheorem{conjecture}[theorem]{Conjecture}
\newtheorem{claim}[theorem]{Claim}

%Book metadata
%\title{A Tufte-Style Book\thanks{Thanks to Edward R.~Tufte for his inspiration.}}
%\author[The Tufte-LaTeX Developers]{The Tufte-LaTeX\ Developers}
%\publisher{Publisher of This Book}

%%
% If they're installed, use Bergamo and Chantilly from www.fontsite.com.
% They're clones of Bembo and Gill Sans, respectively.
%\IfFileExists{bergamo.sty}{\usepackage[osf]{bergamo}}{}% Bembo
%\IfFileExists{chantill.sty}{\usepackage{chantill}}{}% Gill Sans

%\usepackage{microtype}
\usepackage{hyperref}
%%
% Just some sample text

%%
% For nicely typeset tabular material
\usepackage{booktabs}

%%
% For graphics / images
\usepackage{graphicx}
\setkeys{Gin}{width=\linewidth,totalheight=\textheight,keepaspectratio}
\graphicspath{{graphics/}}

% The fancyvrb package lets us customize the formatting of verbatim
% environments.  We use a slightly smaller font.
\usepackage{fancyvrb}
\fvset{fontsize=\normalsize}

%%
% Prints argument within hanging parentheses (i.e., parentheses that take
% up no horizontal space).  Useful in tabular environments.
%\newcommand{\hangp}[1]{\makebox[0pt][r]{(}#1\makebox[0pt][l]{)}}

%%
% Prints an asterisk that takes up no horizontal space.
% Useful in tabular environments.
%\newcommand{\hangstar}{\makebox[0pt][l]{*}}

%%
% Prints a trailing space in a smart way.
\usepackage{xspace}


% Prints the month name (e.g., January) and the year (e.g., 2008)
\newcommand{\monthyear}{%
  \ifcase\month\or January\or February\or March\or April\or May\or June\or
  July\or August\or September\or October\or November\or
  December\fi\space\number\year
}


% Prints an epigraph and speaker in sans serif, all-caps type.
\newcommand{\openepigraph}[2]{%
  %\sffamily\fontsize{14}{16}\selectfont
  \begin{fullwidth}
  \sffamily\large
  \begin{doublespace}
  \noindent\allcaps{#1} \\ % epigraph
  \noindent\allcaps{#2}% author
  \end{doublespace}
  \end{fullwidth}
}
% Generates the index
\usepackage{makeidx}
\makeindex


\makeatletter
\renewcommand{\maketitlepage}{%
\begingroup%
\setlength{\parindent}{0pt}}

{\fontsize{24}{24}\selectfont\textit{\@author}\par}

\vspace{1.75in}{\fontsize{36}{54}\selectfont\@title\par}

\vspace{0.5in}{\fontsize{14}{14}\selectfont\textsf{\smallcaps{\@date}}\par}

\vfill{\fontsize{14}{14}\selectfont\textit{\@publisher}\par}

\thispagestyle{empty}
\endgroup

\makeatother

\titlecontents{part}%
    [0pt]% distance from left margin
    {\addvspace{0.25\baselineskip}}% above (global formatting of entry)
    {\allcaps{Part~\thecontentslabel}\allcaps}% before w/ label (label = ``Part I'')
    {\allcaps{Part~\thecontentslabel}\allcaps}% before w/o label
    {}% filler and page (leaders and page num)
    [\vspace*{0.5\baselineskip}]% after

\titlecontents{chapter}%
    [4em]% distance from left margin
    {}% above (global formatting of entry)
    {\contentslabel{2em}\textit}% before w/ label (label = ``Chapter 1'')
    {\hspace{0em}\textit}% before w/o label
    {\qquad\thecontentspage}% filler and page (leaders and page num)
    [\vspace*{0.5\baselineskip}]% after
%%%% End additional code by Kevin Godby