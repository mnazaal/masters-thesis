% Created 2021-03-05 Fri 21:35
% Intended LaTeX compiler: pdflatex
\documentclass[11pt]{article}
\usepackage[utf8]{inputenc}
\usepackage[T1]{fontenc}
\usepackage{graphicx}
\usepackage{grffile}
\usepackage{longtable}
\usepackage{wrapfig}
\usepackage{rotating}
\usepackage[normalem]{ulem}
\usepackage{amsmath}
\usepackage{textcomp}
\usepackage{amssymb}
\usepackage{capt-of}
\usepackage{hyperref}
\author{nazaal}
\date{}
\title{Notes}
\hypersetup{
 pdfauthor={nazaal},
 pdftitle={Notes},
 pdfkeywords={},
 pdfsubject={},
 pdfcreator={Emacs 26.3 (Org mode 9.4.4)}, 
 pdflang={English}}
\begin{document}

\maketitle
\bibliography{../../Dropbox/org/bibliography/references}

\section{Quick ideas}
\label{sec:org378d690}
\begin{itemize}
\item Topological ordering of nodes in digraphs and connection to topological sort
\end{itemize}


\section{Short notes}
\label{sec:orgd8a5a30}
A fairly good introduction to the baasic terminology is provided in the background paragraph of \cite{sondhi-2019-reduc-pc-algor}. Good blog post available on \href{https://ermongroup.github.io/cs228-notes/}{here}
\section{Key concepts and Definitions}
\label{sec:orge438781}
\subsection{Directed Acyclic Graphs (DAGs)}
\label{sec:orgad4359d}
\begin{itemize}
\item Confounding, explaining away, explanation of why RCTs work well
\item Definitions such as adjacency, parents, children, paths, directed paths, ancestors, descenddants, DAG skeletons, unshielded triples, colliders, unshielded colliders
\item Conditional independence relations and context-specific independence relations
\item Local Markov Property, Markov Equivalence
\item d-separation, m-separation
\item Faithfulness assumption (independence implies no causal relation), strong faithfulness assumption
\item Complete Partially DAG/Essential graph, Interventional Essential graph
\item Maximal Ancestral Graphs and Partial Ancestral Graphs
\item Distribution being causal with respect to a DAG
\end{itemize}
\subsection{Causal discovery on DAGs}
\label{sec:org81de1ee}
\begin{itemize}
\item Search and score based approaches, assuming different causal structures and optimizing a certain score
\item Constraint based appraoches, starting with finding  the conditional independence relations from data then rule out incompatible graphs
\item PC algorithm (constraint based approach)
\item Greedy-Equivalent-Search (GES) (score based approach)
\item Max-Min Hill Climbing  (Mix of both)
\item do-calculus as a means of identification for interventions
\end{itemize}
\subsection{Context-Specific Trees (CSTrees)}
\label{sec:org894d649}
\begin{itemize}
\item Converting a DAG to a CSTree
\item Markov Equivalence for CSTrees
\end{itemize}
\subsection{Causal discovery on CSTrees}
\label{sec:orgeb9458e}

\section{Thesis Sections}
\label{sec:orgbd6c8e6}
\subsection{Introduction}
\label{sec:org3e7a33a}
\subsubsection{Causal models}
\label{sec:org5e62f0c}
\subsubsection{Why they are important}
\label{sec:orgc4673c7}
\subsubsection{Pearl's Hierarchy}
\label{sec:org046d640}
\subsubsection{Causal discovery}
\label{sec:orgc81d120}
\subsubsection{Randomized Controlled Trials}
\label{sec:org3aad5dd}
\subsection{Learning DAGs}
\label{sec:org9712cb4}
\subsubsection{Conditional Independence Relations}
\label{sec:org14398fe}
\subsubsection{DAGs}
\label{sec:orgf9dcd54}
\subsubsection{The PC algorithm}
\label{sec:org5e287c0}
\subsubsection{? Split learning from observational and interventional data}
\label{sec:org9872495}
\subsection{Learning CSTrees}
\label{sec:org68cc0e2}
\subsubsection{Context-Specific Conditional Independence Relations}
\label{sec:org4c9f42a}
\subsubsection{CSTrees}
\label{sec:org6625070}
\subsubsection{Extended PC algorithm}
\label{sec:org76013fd}
\subsubsection{? Split learning from observational and interventional data}
\label{sec:orgfb80962}
\subsection{Experiments}
\label{sec:orgc6c6b35}

\section{Implementation}
\label{sec:org8c279c5}
\subsection{Generate CSTree from DAG, and generate collection of DAGs from CSTrees}
\label{sec:orgad7c748}
\subsection{Running conditional independence tests}
\label{sec:org4694f47}

\section{Org mode details}
\label{sec:orge4cc01e}
\begin{itemize}
\item \LaTeX{} export blocks for things like multi figures, (Caption can still be in org-mode using org-ref), then export them via standalone package and load if the org-export mode is not latex pdf. OR, make a different .tex file for each such items (multifigures, tikz figures etc), compile them using org-babel/shell script, and load as a single image. For external code running experiments, call the APIs within org-babel blocks to separate the code implementation to where they are used.
\end{itemize}

\section{Conventions}
\label{sec:org3bb0e3f}
\begin{itemize}
\item Blackboard bold for probability-theoretic notation, like probabilities and expectations etc
\item Normal bold for graph-theoretic notation, like parents, children etc
\end{itemize}

\section{Doubts}
\label{sec:org903bdce}
\begin{itemize}
\item When we factorize say p(X,Z) into p(X|Z)p(Z) or p(Z|X)p(X) are we saying Z causes X and X causes Z respectively?
\item FCMs and SCMs, definitions look the same, confirm if there is any slight difference
\end{itemize}
\end{document}